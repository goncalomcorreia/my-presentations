\usetheme{dark18}
\usepackage{stmaryrd}
\usepackage{nicefrac}
\usepackage{graphicx}
\usepackage{stackrel}
\usepackage{booktabs}
\usepackage{xspace}
\usepackage{tikz-dependency}
% \usepackage{tkz-graph}
\usepackage{pgfplots}
\usepackage{subcaption}
\usepackage{cancel}
\usepgfplotslibrary{groupplots}
% emojis!
\makeatletter
\newcommand*\fsize{\dimexpr\f@size pt\relax}
\makeatother
\newcommand*\emoji[2][1]{\includegraphics[width=#1\fsize]{emoji/#2}}
%
%% Generic TiKZ utils!
\usetikzlibrary{calc,backgrounds,arrows,arrows.meta}
\usetikzlibrary{decorations.pathreplacing}
\usetikzlibrary{tikzmark,positioning,patterns}
\usetikzlibrary{bending}
\usetikzlibrary{overlay-beamer-styles}
\pgfdeclarelayer{background}
\pgfdeclarelayer{foreground}
\pgfsetlayers{background,main,foreground}
%
\tikzset{
    visible on/.style={alt={#1{}{invisible}}},
    invisible/.style={opacity=0},
    alt/.code args={<#1>#2#3}{%
      \alt<#1>{\pgfkeysalso{#2}}{\pgfkeysalso{#3}} % \pgfkeysalso doesn't change the path
    },
  }

%% particular stuff?
\newcommand*{\ticksize}{2pt}
\tikzset{axisline/.style={thick,myfg!50!mybg,text=myfg,font=\small}}
\tikzset{axislabel/.style={font=\small}}
\tikzset{point/.style={thick,tMidnight}}


% semantic color definitions
\colorlet{colorArgmax}{tTarmac}
\colorlet{colorSoftmax}{tVividBlue}
\colorlet{colorSparsemax}{tYellow!50}
\colorlet{colorFusedmax}{tPeony!80!mybg}
\colorlet{colorEntmax}{tPeony}


\tikzset{
    tinyarrowl/.style={-{Straight Barb[bend]},thick,bend left=75,tPeony},
    tinyarrowr/.style={-{Straight Barb[bend]},thick,bend right=75,tPeony}}
\newcommand{\miniparse}[1]{%
{\footnotesize%
\begin{tikzpicture}[
    node distance=14pt,
    every node/.style={inner sep=0,outer sep=1pt},
    ]
\node[tPeony] (a) at (0, 0) {\textbullet};
\node[tPeony,right of=a] (b) {\textbullet};
\node[tPeony,right of=b] (c) {\textbullet};
\foreach \i/\j/\r/\o in {#1}{
\path (\i.north) edge[tinyarrow\r,opacity=\o] (\j.north);
}\end{tikzpicture}}}

% Tikz pic: draw a cog
\tikzset{cog/.pic={code={
    \draw[myfg,thick,fill=tTarmac!90]
  (0:2)
  \foreach \i [evaluate=\i as \n using {(\i-1)*360/18}] in {1,...,18}{%
    arc (\n:\n+10:2) {[rounded corners=1.5pt] -- (\n+10+2:2.4)
    arc (\n+10+2:\n+360/18-2:2.4)} --  (\n+360/18:2)
  };
  \draw[myfg,thick,fill=mybg] (0,0) circle[radius=.5];
}}}
%
%
% Tikz: draw an envelope
\newsavebox\envelope
\savebox{\envelope}{

\newlength\mylen
\setlength\mylen{2cm}
\begin{tikzpicture}[scale=.65]

\coordinate (A) at (0,0);
\coordinate (B) at (1.41\mylen,-\mylen);
\clip
  ([xshift=-0.5\pgflinewidth,yshift=0.5\pgflinewidth]A) --
  ([xshift=0.5\pgflinewidth,yshift=0.5\pgflinewidth]A-|B) --
  ([xshift=0.5\pgflinewidth,yshift=-0.5\pgflinewidth]B) --
  ([xshift=-0.5\pgflinewidth,yshift=-0.5\pgflinewidth]B-|A) --
  ([xshift=-0.5\pgflinewidth,yshift=0.5\pgflinewidth]A);
\draw[mybg,fill=tTarmac,line cap=rect]
  (A) -- (A-|B) -- (B) -- (B-|A) -- (A);
\draw[mybg]
  (B-|A) -- (0.705\mylen,-.3\mylen) -- (B);
\draw[mybg,fill=tTarmac!90,rounded corners=15pt]
  (A) -- (0.705\mylen,-0.6\mylen) -- (A-|B);
\node[anchor=north]
  at ($ (A)!0.5!(B|-A) $ ) {\parbox{\mylen}{}};
\draw[mybg] (A) -- (B|-A);
\end{tikzpicture}
}
%
% output
\newsavebox\sentoutput
\savebox{\sentoutput}{%
    \begin{tikzpicture}[scale=.75,text=myfg]
 \draw[rounded corners,tTarmac!60,very thick] (-.5, -.5) rectangle (.5, 2.5) {};

 \node[label={[label distance=.25cm]0:positive}] (pos) at (0, 2) {};
 \node[label={[label distance=.25cm]0:neutral}] (neu) at (0, 1) {};
 \node[label={[label distance=.25cm]0:negative}] (neg) at (0, 0) {};

 \draw[fill=tNavy!80!myfg] (pos) circle[radius=9pt];
 \draw[fill=tNavy!20!myfg] (neu) circle[radius=9pt];
 \draw[fill=tNavy!20!myfg] (neg) circle[radius=9pt];
 \end{tikzpicture}
 }
%
% cartoon structure
\newcommand{\cartoon}[2][1]{%
\begin{tikzpicture}%
\node[draw=none, minimum size=#1*1cm, regular polygon, regular polygon sides=5] (p) {};
%
\foreach \i/\j in {#2}%
{
    \draw[tPeony, ultra thick] (p.corner \i) -- (p.corner \j);
}
%
\foreach \i in {1, ..., 5}%
{
    \draw[myfg,fill=mybg,very thick] (p.corner \i) circle[radius=#1*5pt];
}
\end{tikzpicture}}
\newcommand{\cartoonDense}[1][1]{%
\begin{tikzpicture}%
\node[draw=none, minimum size=#1*1cm, regular polygon, regular polygon sides=5] (p) {};
%
\draw[tPeony, ultra thick, opacity=.8] (p.corner 1) -- (p.corner 2);
\draw[tPeony, ultra thick, opacity=.5] (p.corner 1) -- (p.corner 3);
\draw[tPeony, ultra thick, opacity=.7] (p.corner 1) -- (p.corner 4);
\draw[tPeony, ultra thick, opacity=.4] (p.corner 1) -- (p.corner 5);
\draw[tPeony, ultra thick, opacity=.6] (p.corner 2) -- (p.corner 3);
\draw[tPeony, ultra thick, opacity=.3] (p.corner 2) -- (p.corner 4);
\draw[tPeony, ultra thick, opacity=.9] (p.corner 2) -- (p.corner 5);
\draw[tPeony, ultra thick, opacity=.2] (p.corner 3) -- (p.corner 4);
\draw[tPeony, ultra thick, opacity=.6] (p.corner 3) -- (p.corner 5);
\draw[tPeony, ultra thick, opacity=.4] (p.corner 4) -- (p.corner 5);
%
\foreach \i in {1, ..., 5}%
{
    \draw[myfg,fill=mybg,very thick] (p.corner \i) circle[radius=#1*5pt];
}
\end{tikzpicture}}
\newcommand{\cartoonSparse}[1][1]{%
\begin{tikzpicture}%
\node[draw=none, minimum size=#1*1cm, regular polygon, regular polygon sides=5] (p) {};
%
\draw[tPeony, ultra thick, opacity=1] (p.corner 1) -- (p.corner 4);
\draw[tPeony, ultra thick, opacity=.5] (p.corner 2) -- (p.corner 5);
\draw[tPeony, ultra thick, opacity=.5] (p.corner 1) -- (p.corner 5);
%
\foreach \i in {1, ..., 5}%
{
    \draw[myfg,fill=mybg,very thick] (p.corner \i) circle[radius=#1*5pt];
}
\end{tikzpicture}}
%
\newcommand{\setupsimplexbary}[1][3.3]{%
\coordinate (L1) at (0:0);
\coordinate (L2) at (0:#1);
\coordinate (L3) at (60:#1);

\node[label=east:{\small$\triangle$}] at (L2) {};

\fill[tYellow,opacity=.15]  (L1) -- (L2) -- (L3) -- cycle;
\draw[very thick,tYellow] (L1) -- (L2) -- (L3) -- cycle;

\draw[tYellow,fill] (L1) circle[radius=3pt];
\draw[tYellow,fill] (L2) circle[radius=3pt];
\draw[tYellow,fill] (L3) circle[radius=3pt];
}

\newcommand{\drawcs}{%
\node[anchor=south] at (0, \vecheight*4) {$c_1$};
\node[anchor=south] at (0, \vecheight*3) {$c_2$};
\node[anchor=south] at (0, \vecheight*1.5) {$\cdots$};
\node[anchor=south] at (0, \vecheight*0) {$c_N$};}

\newcommand{\drawscores}{%
\node[anchor=south] at (-1-.5*\vecwidth, \vecheight*5+.1) {$\s$};
\draw[elem,fill=vecfg!60!vecbg] (-1-\vecwidth, \vecheight*4) rectangle (-1, \vecheight*5);
\draw[elem,fill=vecfg!85!vecbg] (-1-\vecwidth, \vecheight*3) rectangle (-1, \vecheight*4);
\draw[elem,fill=vecfg!60!vecbg] (-1-\vecwidth, \vecheight*2) rectangle (-1, \vecheight*3);
\draw[elem,fill=vecfg!75!vecbg] (-1-\vecwidth, \vecheight*1) rectangle (-1, \vecheight*2);
\draw[elem,fill=vecfg!50!vecbg] (-1-\vecwidth, \vecheight*0) rectangle (-1, \vecheight*1);
}

\newcommand{\drawnumscores}{%
\node[mygr,anchor=south east] at (-2, \vecheight*4) {\small $2$};
\node[mygr,anchor=south east] at (-2, \vecheight*3) {\small $4$};
\node[mygr,anchor=south east] at (-2, \vecheight*2) {\small $-1$};
\node[mygr,anchor=south east] at (-2, \vecheight*1) {\small $1$};
\node[mygr,anchor=south east] at (-2, \vecheight*0) {\small $-3$};
}

\newcommand{\drawnummax}{%
\node[mygr,anchor=south west] at (2, \vecheight*4) {\small $0$};
\node[mygr,anchor=south west] at (2, \vecheight*3) {\small $1$};
\node[mygr,anchor=south west] at (2, \vecheight*2) {\small $0$};
\node[mygr,anchor=south west] at (2, \vecheight*1) {\small $0$};
\node[mygr,anchor=south west] at (2, \vecheight*0) {\small $0$};
}


\newcommand{\drawargmax}{%
\node[anchor=south] at (1+.5*\vecwidth, \vecheight*5+.1) {$\p$};
\draw[elem,fill=vecfg! 0!vecbg]  (1, \vecheight*4) rectangle (1+\vecwidth, \vecheight*5);
\draw[elem,fill=vecfg!70!vecbg]  (1, \vecheight*3) rectangle (1+\vecwidth, \vecheight*4);
\draw[elem,fill=vecfg! 0!vecbg]  (1, \vecheight*2) rectangle (1+\vecwidth, \vecheight*3);
\draw[elem,fill=vecfg! 0!vecbg]  (1, \vecheight*1) rectangle (1+\vecwidth, \vecheight*2);
\draw[elem,fill=vecfg! 0!vecbg]  (1, \vecheight*0) rectangle (1+\vecwidth, \vecheight*1);
}
\newcommand{\drawsoftmax}{%
\node[anchor=south] at (1+.5*\vecwidth, \vecheight*5+.1) {$\p$};
\draw[elem,fill=vecfg!30!vecbg]  (1, \vecheight*4) rectangle (1+\vecwidth, \vecheight*5);
\draw[elem,fill=vecfg!50!vecbg]  (1, \vecheight*3) rectangle (1+\vecwidth, \vecheight*4);
\draw[elem,fill=vecfg!35!vecbg]  (1, \vecheight*2) rectangle (1+\vecwidth, \vecheight*3);
\draw[elem,fill=vecfg!25!vecbg]  (1, \vecheight*1) rectangle (1+\vecwidth, \vecheight*2);
\draw[elem,fill=vecfg!15!vecbg]  (1, \vecheight*0) rectangle (1+\vecwidth, \vecheight*1);
}


% math and notation
\newcommand*\bs[1]{\boldsymbol{#1}}
\newcommand\defeq{{\,\raisebox{1pt}{$:$}=}\,}
\newcommand\p{\bs{p}}
\newcommand\xx{z}
\newcommand\x{\bs{\xx}}
\newcommand\s{\x}
\renewcommand\ss{\xx}
\newcommand\mg{\bs{\mu}}
\newcommand\pr{\bs{\eta}}
\newcommand\Mp{\mathcal{M}}
\newcommand\parp{\bs{\pi}}
\newcommand\clfp{\bs{\phi}}
\DeclareMathOperator*{\argmax}{arg\,max}
\DeclareMathOperator*{\argmin}{arg\,min}
\DeclareMathOperator{\HH}{H}
\DeclareMathOperator{\mapo}{\bs{\pi}_{\Omega}}
\DeclareMathOperator{\diag}{diag}
\DeclareMathOperator{\ident}{Id}
\DeclareMathOperator{\dom}{dom}
\DeclareMathOperator{\conv}{conv}
\newcommand\reals{\mathbb{R}}

\newcommand{\pfrac}[2]{\frac{\partial #1}{\partial #2}}
\newcommand{\langp}[2]{\textsc{#1}$\shortrightarrow$\textsc{#2}}
\newcommand{\langpb}[2]{\textsc{#1}$\leftrightarrow$\textsc{#2}}
\newcommand*\entmaxtext{entmax\xspace}
\DeclareMathOperator*{\entmax}{\mathsf{\entmaxtext}}
\newcommand*\aentmax[1][\alpha]{\mathop{\mathsf{#1}\textnormal{-}\mathsf{\entmaxtext}}}
\DeclareMathOperator*{\softmaxlight}{\mathsf{softmax}}

% smaller and gray citation
\let\realcitep\citep
\renewcommand*{\citep}[1]{{\color{mygr}\scriptsize\realcitep{#1}}}
\newcommand*{\citeparg}[2]{{\color{mygr}\scriptsize\parencite[][#2]{#1}}}

\usepackage{etoolbox}
\usepackage{fp}
\makeatletter
\newcounter{ROWcellindex@}
\newtoggle{@doneROWreads}
\newcommand\setstackEOL[1]{%
  \ifstrempty{#1}{\def\SEP@char{ }}{\def\SEP@char{#1}}%
  \expandafter\define@processROW\expandafter{\SEP@char}%
}
\newcommand\define@processROW[1]{%
    \def\@processROW##1#1##2||{%
      \def\@preSEP{##1}%
      \def\@postSEP{##2}%
    }%
}
\newcommand\getargs[1]{%
  \togglefalse{@doneROWreads}%
  \edef\@postSEP{\unexpanded{#1}\expandonce{\SEP@char}}%
  \setcounter{ROWcellindex@}{0}%
  \whileboolexpr{test {\nottoggle{@doneROWreads}}}{%
    \stepcounter{ROWcellindex@}%
    \expandafter\@processROW\@postSEP||%
    \expandafter\ifstrempty\expandafter{\@postSEP}{%
      \toggletrue{@doneROWreads}%
    }{}%
    \csedef{arg\roman{ROWcellindex@}}{\expandonce{\@preSEP}}%
  }%
% \narg GIVES HOW MANY ROWS WERE PROCESSED
  \xdef\narg{\arabic{ROWcellindex@}}%
}
\makeatother
\setstackEOL{,}

\usepackage{ifthen}

\newcounter{index}
\newcommand\makearglist[2]{%
  \def\arglist{}%
  \getargs{#1}%
  \setcounter{index}{\narg}%
  \addtocounter{index}{-1}%
  \FPdiv\dTHETA{270}{\theindex}%
  \setcounter{index}{0}%
  \whiledo{\theindex<\narg}{%
    \FPmul\THETA{\theindex}{\dTHETA}%
    \stepcounter{index}%
    \def\thislabel{$\csname arg\roman{index}\endcsname$}%
    \edef\arglist{\arglist -45+\THETA / \thislabel}%
    \ifthenelse{\equal{\theindex}{\narg}}{}{\edef\arglist{\arglist,}}%
  }%
  \FPsub\pointdif{#2}{\argi}%
  \FPsub\DELTA{\argii}{\argi}%
  \FPdiv\NUMticks{\pointdif}{\DELTA}%
  \FPmul\DEGticks{\NUMticks}{\dTHETA}%
  \FPadd\POINTangle{-45}{\DEGticks}%
}

\newcommand\drawdial[2]{%
\makearglist{#1}{#2}%
\begin{tikzpicture}
  \draw[ultra thick] (0,0) circle [radius=0.5];
  \foreach \angle/\label in \arglist
  {
    \draw +(\angle:0.25) -- +(\angle:0.5);
    \draw (\angle:0.85) node {\label};
  }
  \draw[ultra thick] (0,0) -- +(\POINTangle:0.5);
%
\end{tikzpicture}%
}

\tikzset{
  every overlay node/.style={
    draw=black,fill=white,rounded corners,anchor=north west,
  },
}
% Usage:
% \tikzoverlay at (-1cm,-5cm) {content};
% or
% \tikzoverlay[text width=5cm] at (-1cm,-5cm) {content};
\def\tikzoverlay{%
   \tikz[baseline,overlay]\node[every overlay node];
}%